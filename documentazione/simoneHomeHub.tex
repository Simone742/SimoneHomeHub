\documentclass[11pt, a4paper]{article}
\usepackage[italian]{babel} % Imposta l'italiano come lingua del documento

% Pacchetti per una migliore tipografia
\usepackage[T1]{fontenc}
\usepackage[utf8]{inputenc}
\usepackage{lmodern}
\usepackage{microtype}

% Pacchetti per l'aspetto della pagina
\usepackage{geometry}
\geometry{a4paper, margin=2.5cm}

\usepackage{setspace}
\onehalfspacing

% Pacchetti per titoli e intestazioni personalizzate
\usepackage{titlesec}
\titleformat{\section}
{\normalfont\Large\bfseries}
{\thesection}{1em}{}
\titleformat{\subsection}
{\normalfont\large\bfseries\itshape}
{\thesubsection}{1em}{}

\usepackage{fancyhdr}
\pagestyle{fancy}
\fancyhf{}
\fancyhead[L]{\nouppercase{\leftmark}} % Intestazione a sinistra per tutte le pagine
\fancyhead[R]{\thepage}                % Numero di pagina a destra per tutte le pagine
\renewcommand{\headrulewidth}{0.5pt}
\renewcommand{\footrulewidth}{0pt}

% Pacchetti per elementi visivi (opzionale)
\usepackage{graphicx}
\usepackage{xcolor}
\usepackage{svg}

% Ipercollegamenti (opzionale)
\usepackage{hyperref}
\hypersetup{
    colorlinks=true,
    linkcolor=blue,
    citecolor=blue,
    urlcolor=blue
}

% Pacchetti per elementi visivi (opzionale)
\usepackage{graphicx}             % Inserimento di immagini
\usepackage{xcolor}               % Gestione dei colori

% Ipercollegamenti (opzionale)
\usepackage{hyperref}
\hypersetup{
    colorlinks=true,
    linkcolor=blue,
    citecolor=blue,
    urlcolor=blue
}


\title{Simone Home Hub}
\author{Simone Arena}
\date{\today}

\begin{document}

\maketitle

\begin{figure}[h!]
    \centering
    \includegraphics[width=0.2\textwidth]{media/Home_Assistant_logo_(2023).svg}
    \hspace{1cm} % Spazio tra le immagini
    \includegraphics[width=0.2\textwidth]{media/esphome-logo.jpeg}
    \caption{A sinistra il logo di Home Assistant, a destra il logo di ESPHome}
    \label{fig:logos}
\end{figure}

\newpage

\tableofcontents

\newpage

\section{Metanote sul documento}
Questo documento è stato redatto in LaTeX, un linguaggio di markup per la scrittura di documenti scientifici e tecnici.
LaTeX è particolarmente utile per la creazione di documenti complessi, come tesi, articoli scientifici e relazioni tecniche, 
grazie alla sua capacità di gestire formule matematiche, bibliografie e riferimenti incrociati in modo efficiente.

\newpage

\section{Introduzione}
Il presente progetto si inserisce nel dinamico 
e in continua evoluzione panorama dell'Internet 
delle Cose (IoT) e dell'analisi dei Big Data. 
L'IoT rappresenta una fitta rete di dispositivi 
fisici interconnessi, capaci di raccogliere e 
scambiare dati in tempo reale. Questa mole crescente 
di informazioni, definita come Big Data, offre opportunità 
senza precedenti per comprendere e ottimizzare il nostro ambiente, 
in particolare all'interno delle nostre abitazioni. 
In questo contesto, esploreremo le potenzialità di una 
piattaforma open-source di home automation come Home Assistant, 
focalizzandoci su come essa possa agire da fulcro per l'integrazione 
di diversi dispositivi intelligenti e sulla gestione dei dati generati, 
aprendo la strada a soluzioni innovative per una casa più efficiente, 
sicura e personalizzata.

\section{Obiettivi}
L'obiettivo ultimo nel progetto non è propriamente definito, in quanto,
trattandosi di una tecnologia 
in continua evoluzione, sarebbe difficile
definire un traguardo 
finale. Tuttavia, si 
possono definire alcune milestones
intermedie e avanzate, 
che possono essere raggiunge in un tempo ragionevole,
arrivando ad un risultato soddisfacente.
In particolare, il progetto in questo caso si propone di creare un sistema
di automazione del mio piccolo orto, rendendo automatica l'irrigazione delle piantine.
Le milestones intermedie sono:
\begin{itemize}
    \item Setup del server Home Assistant
    \item Integrazione dei primi dispositivi domotici
    \item Creazione di automazioni di base
    
\end{itemize}

Le milestones avanzate sono:
\begin{itemize}
    \item Integrazione di dispositivi avanzati (telecamere, sensori di movimento, ecc.)
    \item Creazione di automazioni complesse
    \item Analisi dei dati generati dai dispositivi
    \item Posizionamento di un sensore di umidità del terreno
    \item Integrazione di un sistema di irrigazione automatica
    \item Strutturazione definitiva della dashboard
\end{itemize}

\section{Introduzione}

\subsection{Home Assistant}

\subsubsection{Cos'è Home Assistant}
\textit{Home Assistant} è una piattaforma di automazione domestica. È gratis e open-source, e
rappresenta un'alternativa completamente locale a servizi come HomeBridge e SmartThings.

\subsubsection{Perchè Home Assistant}
Home Assistant rende possibile l'automazione domestica senza la necessità di un cloud,
quindi senza dipendere dall'infrastuttura internet, da server remoti o da servizi esterni.
Questo rende l'esperienza utente più fluida e l'operabilità piu affidabile.
Un sistema basato su Home Assistant è di sua natura un sistema DIY. La sua vasta compatibilità
con i dispositivi è dovuta proprio al fatto che ogni sensore o microcontrollore può essere integrato
deve essere configurato e programmato individualmente. Ciò permette numerose possibilità di
personalizzazione: ogni sistema può essere adattato alle esigenze specifiche dell'utente.

\subsubsection{Perché preferire HAss alle soluzioni cloud-based}
La grande maggioranza dei dispositivi smart è progettata per essere utilizzata basandosi su
un'infrastuttura basata sul web, che spesso rende più facile e immediato il setup iniziale per gli utenti,
ma anche più capillare il controllo dei dati da parte dei produttori, come la raccolta e l'analisi dei dati
raccolti dai sensori domotici. Questo è un aspetto che non deve essere sottovalutato, in quanto
l'analisi dei dati è una pratica sempre più diffusa, che sta permettendo alle case produttrici di
ottimizzare i propri prodotti, ma se non gestita correttamente, può portare a violazioni della privacy
e alla diffusione di dati sensibili.

\subsection{ESPHome}

\subsubsection{Cos'è ESPHome}
\href{https://esphome.io}{\textit{ESPHome}} è un framework \textbf{open-source} che permette di creare firmware personalizzati per dispositivi basati su ESP8266, ESP32 e RP2040,
che sono ampiamente utilizzati in progetti di automazione domestica e IoT grazie alla loro versatilità e facilità d'uso.

\subsubsection{Perché ESPHome}
\begin{itemize}
    \item In questo progetto verranno utilizzati dispositivi basati su ESP8266 e ESP32;
    \item ESPHome semplifica la configurazione e la programmazione di questi dispositivi, 
    consentendo agli utenti di definire il comportamento del dispositivo attraverso un file di configurazione YAML.
    In questo file, gli utenti possono specificare le funzionalità desiderate, come sensori, attuatori e altre componenti hardware. 
    Una volta configurato, ESPHome genera automaticamente il firmware necessario per il microcontrollore, che può essere caricato direttamente sul dispositivo;
    \item Questo approccio consente di risparmiare tempo e fatica nella scrittura del codice, rendendo l'automazione domestica più accessibile anche a chi non ha esperienza di programmazione;
    \item ESPHome offre un'integrazione fluida con Home Assistant, consentendo agli utenti di monitorare e controllare i dispositivi direttamente dalla dashboard.

\end{itemize}

\section{Sviluppo del progetto}

\subsection{Setup del server}
Per l'hosting di Home Assistant di questo progetto è stato scelto un server dedicato.
Il server in questione non è altro che un computer da ufficio datato ormai in disuso, basato su
piattaforma x86.
In questo caso Home Assistant sarà installato come un vero sistema operativo, che verrà esclusivamente
gestito dal computer dedicato.

\subsubsection{Preparazione della macchina}
È fondamentale che la macchina dedicata sia in buone condizioni, che non presenti problemi hardware e che
i componenti al suo interno siano orientati al risparmio energetico. Essendo un computer destinato
a rimanere acceso 24 ore su 24, è importante che al suo interno sia munito di adeguati sistemi di raffreddamento.

\subsubsection{Scelta dell'hardware}
La scelta dell'hardware è fondamentale per garantire prestazioni ottimali e un funzionamento affidabile del sistema.
Per l'installazione del sistema operativo è richiesto un computer con almeno 2 GB di RAM e 32 GB di spazio di archiviazione.
Home Assistant OS può essere installato sia su SBC, come un Raspberry Pi, che su una comune macchina x86.
Nella scelta del supporto di memoria, è preferibile optare per un'unità a stato solido (SSD) piuttosto che un disco rigido tradizionale (HDD),
in quanto, oltre alle prestazioni nettamente superiori, gli SSD non prevedono parti meccaniche in movimento, non sono quindi
soggetti a usura meccanica e spesso consumano meno energia.

\subsubsection{Installazione del sistema operativo}
In questo caso, il sistema operativo scelto è \textit{Home Assistant Operating System}, che è una distribuzione Linux
ottimizzata per eseguire Home Assistant, che non prevede un'interfaccia grafica, ma è gestita tramite un'interfaccia web.
Questo sistema operativo è progettato per essere installato su hardware dedicato, come un Raspberry Pi o un server x86, 
e offre un'installazione semplificata e una gestione centralizzata dei dispositivi e delle automazioni.
L'installazione di Home Assistant OS è un processo relativamente semplice.
\begin{enumerate}
    \item Scaricare l'immagine del sistema operativo dal sito ufficiale di Home Assistant;
    \item Creare un'unità USB avviabile utilizzando un software come Balena Etcher o Rufus;
    \item Avviare il computer dal supporto USB e seguire le istruzioni per completare l'installazione.
    \item Una volta completata l'installazione, il sistema si riavvierà e Home Assistant sarà accessibile tramite un browser web all'indirizzo IP del server.
\end{enumerate}

Solamente durante questo processo sarà necessario un monitor e una tastiera, che serviranno per interagire con l'interfaccia
a testo per la primissima configurazione.

\subsubsection{Posizionamento del server}
Una volta completata l'installazione, il server può essere posizionato in un luogo strategico all'interno della casa,
possibilmente raggiunto da una rete cablata, per garantire una connessione stabile e veloce.
Ad ogni modo, il server può essere posizionato in un luogo remoto, purché sia raggiunto da una rete Wif-Fi.
L'importanche è che la scelta del posizionamento sia fatta in modo da garantire la sicurezza della macchina, che deve
essere posta su un supporto stabile e sicuro, lontana da oggetti infiammabili, da fonti di calore e da umidità.

\subsection{Strutturazione della rete}
Per poter garantirei funzionamento dei microcontrollori all'interno dell'orto, è 
necessaria una copertura Wi-Fi adeguata, quindi di conseguenza bisognerà procedere
all'acquisto e al posizionamento di un access point Wi-Fi, che andrà cablato.

\subsubsection{Scelta dell'access point}
In questo caso, è sufficiente qualsiasi access point, purché sia compatibibile con
la frequenza di 2.4 GHz, che è la frequenza più comunemente utilizzata dai dispositivi IoT e
dagli ESP32. Per motivi quasi esclusivamente economici è stato scelto il Tenda F3.

\subsubsection{Configurazione dell'access point}
Di default, il Tenda F3 è impostato come un router. Per poterne fare l'uso di cui
necessitiamo, sarà quindi necessario procedere alla configurazione, collegando
il proprio computer all'AP tramite cavo Ethernet e accedendo alla sua dashboard
tramite l'indirizzo IP di default fornito dal costruttore.
In questo caso è stato scelto di creare una rete Wi-Fi separata, con un ssid diverso
da quella domestica, anche se all'access point è stato dato un indirizzo IP appartenente
alla rete di casa. La manzione del DHCP è stata delegata al router.

\subsubsection{Posizionamento dell'access point}
È fondamentale che l'access point sia ben posionato per garantire una copertura omogenea
su tutta la superficie dell'orto e quindi non creare limitazioni, permettendo di posizionare
microcontrollori Wi-Fi liberamente. Se l'AP viene posizionato esternamente ed
il suo modello non è stato concepito per trovarsi all'aria aperta, esposto
alle intemperie e al clima, è fortemente consigliato posizionarlo all'interno di un
contenitore ermetico.

\subsubsection{Cablaggio dell'access point}
Nel mio caso, l'orto si trova su un lato della mia casa dal quale sono presenti diversi
corrugati che partono dall'interno. Per questo, una volta posizionato l'AP sono stati
fatti passare grazie a una sonda sia il cavo Ethernet proveniente da uno switch collegato
al router domestico, sia il cavo di alimentazione.



\section{Conclusioni}

\end{document}